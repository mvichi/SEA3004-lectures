%% LyX 2.3.6 created this file.  For more info, see http://www.lyx.org/.
%% Do not edit unless you really know what you are doing.
\documentclass[a4paper,english]{scrartcl}
\usepackage[T1]{fontenc}
\usepackage[latin9]{inputenc}
\usepackage{amsmath}

\makeatletter

%%%%%%%%%%%%%%%%%%%%%%%%%%%%%% LyX specific LaTeX commands.
\pdfpageheight\paperheight
\pdfpagewidth\paperwidth


\makeatother

\usepackage{babel}
\usepackage{listings}
\renewcommand{\lstlistingname}{Listing}

\begin{document}
\title{SEA3004F M1P1: Matrix manipulation}
\author{Ocean and Atmosphere Dynamics}
\maketitle

\section{Tutorial on matrix operations and multiplication}

\subsection{Matrix syntax\label{subsec:Matrix-syntax}}

\textbf{The first component of the assignment is to practice the examples
shown in Chapter 4 from Moler's textbook. }This will help you to familiarize
with the syntax and the matrix operations done in class. Download
the zip file and extract it in a local folder named M1P1. 

There is a script that contains all the commands (\texttt{matrices\_recap.m}).
When you open it, you will be asked to convert it to a \emph{livescript}.
We will explain in class what it means and how they work. You used
livescripts when practicing the matlab tutorial. 

There is an advantage of having a command line, because you can test
commands and check the output. Do not copy and paste the commands,
but type them, to familiarize with the syntax. You can use this script
during the week to learn at your own pace how to create matrix objects
and manipulate them (the paragraph on the determinant is not part
of the syllabus, but you are welcome to practice it). 

\subsection{Rotation}

Matrix-matrix or matrix-vector multiplications are introduced through
the rotation of plane figures using a 2x2 matrix. This is an excerpt
from the lecture notes to refresh the rotation matrix definition:
\begin{itemize}
\item rotation by an angle $\theta$ is a \emph{linear transformation} and
it can be expressed as a matrix product
\item {\small{}we can derive the transformation matrix from the transformation
of the coordinate bases $I_{2}$
\[
\mathbf{R_{\theta}}=\left[rot_{\theta}\left(\begin{bmatrix}1\\
0
\end{bmatrix}\right)\ rot_{\theta}\left(\begin{bmatrix}0\\
1
\end{bmatrix}\right)\right]
\]
}{\small\par}
\item {\small{}The generic form of a rotation matrix is
\[
\mathbf{R_{\theta}}=\begin{bmatrix}\cos\theta & -\sin\theta\\
\sin\theta & \cos\theta
\end{bmatrix}
\]
}{\small\par}
\item The rotation of a vector $\mathbf{v}_{0}$ is $\mathbf{v}'=\mathbf{R}_{\theta}\mathbf{v}_{0}$
\end{itemize}
In order to create the matrix in octave/MATLAB, you need to know the
syntax.

The script \texttt{house.m} contains a matrix of point coordinates
that are shaped like a house. Open it in the editor and check the
syntax of functions. You can inspect the matrix simply by calling
the function \texttt{house} (it is a \emph{function}, so it returns
another object as output, which is the matrix). You can plot the matrix
joining the dots using the script \texttt{dot2dot} giving the matrix
as input argument and then print the figure.

Have a look at what \texttt{house} and \texttt{dot2dot} do by calling
the editor:
\begin{lstlisting}[language=Matlab,frame=lines]
>> edit house.m
>> edit dot2dot.m
\end{lstlisting}

You can rotate the house by applying a matrix-matrix multiplication
with a rotation matrix using the script \texttt{wiggle} or \texttt{wiggle\_o}
(if you use octave). These are all the commands to be tested (the
output is not shown)

\begin{lstlisting}[language=Matlab,frame=lines]
>> house
>> figure
>> dot2dot(house)
>> print -dpng HOUSE.png
>> wiggle(house)
\end{lstlisting}

There is a button on the figure to close it. When executed in the
livescript, it will generate an external figure, and then the last
frame will be included in your livescript.

In MATLAB and octave, you can create a new figure and save its content
with the following commands

\begin{lstlisting}[language=Matlab,frame=lines]
>> figure(1)
 % ... plot commands ...
>> print -f1 -dpng FIGURENAME.png
\end{lstlisting}

To create another figure, change the \emph{handle} number in the argument
of the \texttt{figure} function (for instance, \texttt{figure(2)}).

\section{Assignment: Rotation of a wind vector \label{sec:How-to-present}}

Write the code to perform the following operations. 
\begin{enumerate}
\item create the matrix $\mathbf{V}$ holding the point coordinates (-1,-5),
(-1,5), (-3,5), (0,8), (3,5), (1,5), (1,-5) to create a complete drawing
of a wind vector
\item plot the matrix with \texttt{dot2dot} and save the figure in PNG format
\item Create a rotation matrix $\mathbf{G}$ for the angle $\theta=-\pi+\pi/6$
\item Apply the rotation to matrix $\mathbf{V}$ to obtain the new matrix
$\mathbf{W=GV}$, plot the result with \texttt{dot2dot} and save the
figure in PNG format
\end{enumerate}
\textbf{Submit the livescript contaning the same number of sections
as in this document and the figures. There must be comments explaining
what each piece of code does (it is mostly for you, so you will be
able to come back again and study for the final exam).}
\end{document}
